\documentclass[11pt]{article}
\usepackage{fullpage}
\usepackage{enumitem}
\usepackage{amsmath,amssymb}

\title{Debugging Probabilistic Programs}
\author{Vikash K. Mansinghka, Ben Zinberg}

\newcommand{\Xsp}{X_{\text{specific}}}

\begin{document}

\maketitle

\section{Points}

Some points, once brought to your attention, are quite believable a priori:

\begin{itemize}
  \item {\bf Probabilistic program development is not well, sometimes
    incorrectly, understood.}  It's a new field.  Many aspects of conventional
    programming are useful, but many of the basic intuitions from that area can
    lead us astray here.

  \item {\bf Debugging probabilistic programs is like debugging hardware.} While
    probabilistic programs can produce conventional errors, debugging is
    typically about chasing down other kinds of problems.  The output of a
    probabilistic model (or the result of inference on a model) is a probability
    distribution, which can be inspected in a number of ways.  In order to
    ensure that the distribution achieved is acceptable, it is crucial to {\em
    use} many or all of those ways.  It's easy to forget to do this, since in
    the finished product only a neat little corner of the model will be
    inspected.  But as in hardware debugging, it is best to proceed with
    ruthless skepticism, as each component is not just liable but quite likely
    to fail.
    
    In conventional programming (partly due to your trained intuitions as a
    programmer), you can reliably analyze most short chunks of code in your
    head.  You (and we) cannot do this for probabilistic code, both because you
    are not trained and because it is harder.  Also, it is typically much easier
    to sanity-check a short conventional program; in order to sanity-check a
    probabilistic program, you need to inspect a distribution, which requires
    many probes.
\end{itemize}

\section{Steps}

Here is a series of steps proposed by VKM to debug probabilistic programs.  Here
\begin{itemize}[noitemsep]
  \item $X$ is latents
  \item $\lambda$ is parameters
  \item $Y$ is data
  \item $X \mapsto T_Y(X)$ is the ``step'' operator of an inference program,
    such that the true posterior $P(X|Y,\lambda)$ is invariant under $T_Y$, and
    $T_Y^t(X)$ converges to the true posterior $P(X|Y,\lambda)$ as $t \to
    \infty$.
\end{itemize}

Here are the steps:
\begin{enumerate}
  \item ``Do I know the problem I'm asking the prob. prog. to solve?  Do I know
    it's solvable?  Can I solve it?''

    You should not expect v0.1 of your program to be better at solving the
    problem than you---you have eyes and a brain.  Your first test case should,
    a fortiori, be one in which it is completely obvious to you what the
    solution is, both because you need a much stupider program to work, and
    because you need to be able to reliably tell whether the program has
    succeeded.  Along that vein, it is important at this early stage to set up
    adequate {\em visualization} for answers $X$ produced by the program.  Your
    eyes are much more powerful signal processors than your brain's abstract
    number sense, and your inspection plan (you have one, right?) should try to
    leverage that.  Sometimes it will be nice to have visualization of $X$
    before you even have a prior.  That way, each of the (probably many) times
    you rewrite the prior, you can sanity-check by writing a quick loop to
    sample the prior and generate several \texttt{png}s.

  \item ``Do I really grok $P(X|\lambda)$?''

    Cycling through summary \texttt{png}s of samples is a good first step, but
    to go beyond what is visible at first glance, it will be useful to come up
    with a small (or large) portfolio of statistics that can be plotted or
    inspected to reveal more information about the distribution.  Statistics for
    a PCFG could include
    \begin{itemize}[noitemsep]
      \item number of variables in parse tree
      \item number of times each production was used
      \item yield at 0, i.e., the terminal reached by walking left maximally
    \end{itemize}

  \item\label{itm:grok-transition-operator}
    ``Do I grok $T_Y(X)$ at all?''

    It is quite hard to analyze this abstractly by looking at text.  Visual
    methods are necessary:
    \begin{itemize}[noitemsep]
      \item Make a movie of $T_Y^t(X)$, $t=0,1,\ldots$.
      \item Make a movie of Gibbs sampling: an animation of $X^t$, $t=0,1,\ldots$, where
        \[
        X^0 \sim P(X|\lambda), \quad
        Y^t \sim P(Y|X^t,\lambda), \quad
        X^{t+1} \sim T_Y(X^t).
        \]
        To better understand $T$ better, it can be helpful to also do the above
        steps when $Y$ is empty, i.e., when inference is done on no data.  The
        true posterior in this case is trivial to write down, but you are not
        trying to learn about the true posterior, you are trying to learn about
        the inference algorithm.
    \end{itemize}

  \item ``Does $T_Y(X)$ respect Bayes' rule?''

    One way to check this is to run step \ref{itm:grok-transition-operator} with
    $Y$ empty, and compare the empirical distribution of $\{X^t\}$ to the prior
    $P(X|\lambda)$, perhaps using a Q-Q plot on summary statistics.  If $T$ is
    meant to sample the posterior (rather than do MAP estimation or something
    like that) then the distributions should match.

    Another pair of distributions to compare is the history when $Y$ contains
    data versus the distribution inferred by rejection sampling.

  \item ``Now I:
    \begin{enumerate}
      \item Know the problem is solvable (i.e., the data is ok)
      \item Grok my model (to some extent)
      \item Grok my inference program (to some extent)
      \item Grok their interaction (to some extent)
      \item Have inspection infrastructure for model, data, and $T_Y$
      \item Have quantitative evidence in support of intuition
      \item Have (anecdotally) established asymptotic soundness of $T_Y$, so I
        can get `Bayesian' behavior in some regime''
    \end{enumerate}

  \item Determine how much data and inference are needed to ``solve'' random
    instances, and specific instances, and how inference performance compares to
    baselines.

    The idea is to get some measure of how inference quality depends on number
    of steps $t$ and number of data points $N$.  (Or, fixing a certain quality
    goal, how large $t$ has to be as a function of $N$ to achieve that goal, or
    vice versa.)  If you can codify one or several concrete ways to measure
    inference quality for your problem, that would be useful.  One basic/rough
    way to do this is to generate $X^*$ from the prior, $Y^* \sim
    P(Y|X^*,\lambda)$, and double $t$ and $N$ until the inferred posterior
    concentrates on $X^*$, though of course, this effect could be more
    conspicuous or less conspicuous depending on the inference problem.

  \item\label{itm:seatbelt}
    Try to find ``seatbelt variables'' $\{\Xsp\}$ which capture the quintessence
    of the sorts of inference dynamics that occur: something like an orthogonal
    basis for the possible states $X$ which is orthogonal with respect to the
    qualitative inference behaviors produced.

  \item Use step \ref{itm:seatbelt} as a harness to perform full
    diagnostics---i.e., histories/traces of ``score,'' probe statistics,
    marginals, values of $\Xsp$'s, etc.--- on a few ``extreme'' instances (some
    causing failure, either inductive or computational).

  \item Try the real problem---or, add ``?? variable'' [not sure what this means
    -- BZ] and repeat
\end{enumerate}

\section{Notes}

Notes about these steps:
\begin{itemize}
  \item Each of these steps will be run many times.  It is well worth it to set
    up very clean, automatic machinery for the steps.
  \item VKM points out, ``If the model and data generator are both parametrized
    by the same scaling parameters (amount of evidence, number of random
    variables) and if the inference algorithm is parametrized by $T$, then the
    steps are essentially mechanical.''
\end{itemize}

\end{document}
